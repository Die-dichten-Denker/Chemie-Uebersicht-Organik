\documentclass[a4paper]{article}

\usepackage{times}
\usepackage[ngerman]{babel}
\usepackage[T1]{fontenc}
\usepackage{amsmath}
\usepackage{amsfonts}
\usepackage{graphicx}
\usepackage{authblk}
\usepackage{chemfig}
\usepackage{fancyhdr}

\pagestyle{fancy}
\renewcommand{\headrulewidth}{0,6pt}
\fancyhead[L]{\leftmark}
\fancyhead[R]{\thepage}

\title{\Huge{Organische Chemie\\Lernzettel}}
\date{\today}
\author{\quad\\Baden, Julian\\Hagemann, Florian\\\quad\\Gymnasium Mellendorf\\ABI Jahr 2027}

\begin{document}

\maketitle
\thispagestyle{empty}
\newpage
\tableofcontents \thispagestyle{empty}
\newpage
\pagenumbering{arabic}

%________________________________________________________________________________________________


\section{Stoffklassen}
\subsection{Alkane}

Alkane Sind eine Reihe \textbf{Kohlenstoffatome}, mit anliegenden
\textbf{Wasserstoffatomen}. Die Reihe, welche die Alkane bilden, wenn man sie nach der Anzahl C-Atom
ordnet, heißt "homologe Reihe"\\[0,5cm]
\begin{center}
\chemfig{H-C(-[:90]H)(-[:270]H)-C(-[:90]H)(-[:270]H)-H} \hspace{1cm} \textbf{Ethan}\\[0,5cm]
\begin{tabular}{|c|c|c|}
    \hline
    Name        & Molekülformel          & Halbstrukturformel\\ \hline 
    Methan      & \chemfig{C H_4}        & \chemfig{CH_3}\\
    Ethan       & \chemfig{C_2 H_6}      & \chemfig{CH_3 - CH_3}\\
    Propan      & \chemfig{C_3 H_8}      & \chemfig{CH_3 - CH_2 - CH_3}\\
    Butan       & \chemfig{C_4 H_10}     & \chemfig{CH_3 - C_2 H_2 - CH_3}\\
    Pentan      & \chemfig{C_5 H_12}     & \chemfig{CH_3 - C_3 H_4 - CH_3}\\
    Hexan       & \chemfig{C_6 H_14}     & \chemfig{CH_3 - C_4 H_6 - CH_3}\\
    Heptan      & \chemfig{C_7 H_16}     & \chemfig{CH_3 - C_5 H_8 - CH_3}\\
    Octan       & \chemfig{C_8 H_18}     & \chemfig{CH_3 - C_6 H_{10} - CH_3}\\
    Nonan       & \chemfig{C_9 H_20}     & \chemfig{CH_3 - C_7 H_{12} - CH_3}\\
    Decan       & \chemfig{C_{10} H_22}  & \chemfig{CH_3 - C_8 H_{14} - CH_3}\\ \hline
\end{tabular}\\[1cm]
\end{center}

Innehalb der Homologen Reihe sind folgende Zusammenhänge zu erkennen:\\
\begin{itemize}
    \item {Viskousität steigt}
    \item {Siede- \& Schmelztemperatur steigt}
    \item {Dichte nimmt zu} \\
\end{itemize}
Diese Zusammenhänge liegen an steigender Intensität von London- / Van-der-Waals-kräfte mit steigender Kettenlänge.\\[0,5cm]

Alkane besitzen folgene Eigenschaften:\\
\begin{itemize}
    \item{keine elektrische Leitfähigkeit}
    \item {sie sind unpolar}\\[1cm]
\end{itemize}
\newpage


\subsection{Halogenalkane}

Halogenalkane sind Alkane, denen durch \textbf{elektrophile Addition}, aus Alkenen, oder durch
\textbf{radikalische Substitution}, aus Alkanen, ein Halogen addiert wurde. \\

\begin{center}
    \chemfig{H-C(-[:90]H)(-[:270]Br)-C(-[:90]H)(-[:270]H)-H} \hspace{1cm} \textbf{Bromethan}\\[0,5cm]
\end{center}

Halogenalkane sind \textbf{lipophil}, ihre \textbf{Siedetemperatur ist höher als bei Alkanen}.
Bei Mehrfachsubstutution / Mehrfachaddition werden die Halogenalkane \textbf{mit steigender Halogenanzahl reaktionsträger}.\\

Eine Nachweisreaktion für Halogenalkane ist die Beilsteinprobe.

\subsection{Alkene \& Alkine}

Alkene sind Kohlenwasserstoffe, welche eine Doppelbindung zwischen zwei C-Atomen besitzen.
Sie wie die Alkane benannt, besitzen aber eine \textbf{-en} Endung. Vor dieser Endung wird die Stelle der Mehrfachbindung geschrieben,
z.B.: Pent-2-en.\\

\begin{center}
    \chemfig{C (-[:240]H)(-[:120]H) = C (-[:60]H)(-[:300]H)} \hspace{1cm} \textbf{Ethen}\\[0,5cm]
\end{center}


Alkine bekommen hingegen die Ändung \textbf{-in}.\\

\begin{center}
    \chemfig{H - C ~ C - H} \hspace{1cm} \textbf{Ethin}\\[0,5cm]
\end{center}

Wichtig bei Mehrfachsubstutution und Mehrfachaddition ist die Benennung mit cis- und trans-.\\
\begin{center}
    \chemfig{H-C(-[:90]H)(-[:270]Br)=C(-[:90]Br)(-[:270]H)-H} \hspace{1cm} \textbf{Trans-1,2-dibromethen}\\[0,5cm]
    \chemfig{H-C(-[:90]H)(-[:270]Br)=C(-[:90]H)(-[:270]Br)-H} \hspace{1cm} \textbf{Cis-1,2-dibromethen}\\[0,5cm]
\end{center}

Eine Nachweisreaktion für Mehrfachbindungen ist die Entfärbung von Bromwasser.


\subsection{Alkohole}

Alkohole sind Moleküle mit einer O-H-Gruppe (Hydroxygruppe). Sie entstehen durch die
\textbf{nucleophile Substitution} von Halogenalkanen.
Dlol kek

\end{document}