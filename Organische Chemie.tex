\documentclass[a4paper]{article}

\usepackage{times}
\usepackage[ngerman]{babel}
\usepackage[T1]{fontenc}
\usepackage{amsmath}
\usepackage{amsfonts}
\usepackage{graphicx}
\usepackage{chemfig}
\usepackage{mhchem}
\usepackage{fancyhdr}
\usepackage{xcolor}

\pagestyle{fancy}
\renewcommand{\headrulewidth}{0,6pt}
\fancyhead[L]{\leftmark}
\fancyhead[R]{\thepage}

\title{\Huge{Organische Chemie\\Lernzettel}}
\date{\today}
\author{\quad\\Baden, Julian\\Hagemann, Florian\\\quad\\Gymnasium Mellendorf\\ABI Jahr 2027}

\begin{document}

\maketitle
\thispagestyle{empty}
\newpage
\tableofcontents \thispagestyle{empty}
\newpage
\pagenumbering{arabic}

%________________________________________________________________________________________________


\section{Stöchiometrie}
Stöchiometrie, auch chemisches Rechnen, ist ein Werkzeug der Chemie, mit welchen man Stoffmengen verhältnisse
nutzen kann, um zum Beispiel Produktmengen oder Volumen zu errechnen.
\subsection{Größen}
\begin{center}
\begin{tabular}{|c|c|c|}
    \hline
    Größe       &Symbol     &Einheit \\\hline
    Atomanzahl  &$N$        &-- \\ 
    Avogrado-Konstante &$N_a$ &$mol^{-1}$ \\
    Stoffmenge &$n$ &$mol$ \\
    Masse &$m$ &$kg$ \\
    Molare Masse &$M$ &$kg \cdot mol^{-1}$ \\
    Volumen &$V$ &$l$ \\
    Molares Volumen &$V_m$ &$l \cdot mol^{-1}$\\ \hline
\end{tabular}
\end{center}

\subsection{Formeln}
\Large
\begin{eqnarray*}
    n &= \dfrac{N}{N_a} \\ [5mm]
    n &= \dfrac{m}{M} \\ [5mm]
    n &= \dfrac{V}{V_m}
\end{eqnarray*}
\normalsize

\subsection{Standartbedinungen und Normbedingungen}
\begin{center}
    \begin{tabular}{|c|c|c|}
        \hline
        Größen &Standartbedingungen &Normbedingungen \\\hline
        Temperatur &$298\;K = 25\;C$ &$273\;K = 0\;C$ \\
        Druck &$1000 \; hPa$ &$1013 \; hPa$ \\
        Molares Volumen &$V_m = 24 \; l \cdot mol^{-1}$ &$V_{mn} = 22,4 \; l \cdot mol^{-1}$ \\\hline
    \end{tabular}
\end{center}


\newpage

\section{Stoffklassen}
\subsection{Alkane}

Alkane Sind eine Reihe \textbf{Kohlenstoffatome}, mit anliegenden
\textbf{Wasserstoffatomen}. Die Reihe, welche die Alkane bilden, wenn man sie nach der Anzahl C-Atom
ordnet, heißt "homologe Reihe"\\[0,5cm]
\begin{center}
\chemfig{H-C(-[:90]H)(-[:270]H)-C(-[:90]H)(-[:270]H)-H} \hspace{1cm} \textbf{Ethan}\\[0,5cm]
\begin{tabular}{|c|c|c|}
    \hline
    Name        & Molekülformel          & Halbstrukturformel\\ \hline 
    Methan      & \chemfig{C H_4}        & \chemfig{CH_3}\\
    Ethan       & \chemfig{C_2 H_6}      & \chemfig{CH_3 - CH_3}\\
    Propan      & \chemfig{C_3 H_8}      & \chemfig{CH_3 - CH_2 - CH_3}\\
    Butan       & \chemfig{C_4 H_10}     & \chemfig{CH_3 - C_2 H_2 - CH_3}\\
    Pentan      & \chemfig{C_5 H_12}     & \chemfig{CH_3 - C_3 H_4 - CH_3}\\
    Hexan       & \chemfig{C_6 H_14}     & \chemfig{CH_3 - C_4 H_6 - CH_3}\\
    Heptan      & \chemfig{C_7 H_16}     & \chemfig{CH_3 - C_5 H_8 - CH_3}\\
    Octan       & \chemfig{C_8 H_18}     & \chemfig{CH_3 - C_6 H_{10} - CH_3}\\
    Nonan       & \chemfig{C_9 H_20}     & \chemfig{CH_3 - C_7 H_{12} - CH_3}\\
    Decan       & \chemfig{C_{10} H_22}  & \chemfig{CH_3 - C_8 H_{14} - CH_3}\\ \hline
\end{tabular}\\[1cm]
\end{center}

Innehalb der Homologen Reihe sind folgende \textbf{Zusammenhänge} zu erkennen:\\
\begin{itemize}
    \item {Viskousität steigt}
    \item {Siede- \& Schmelztemperatur steigt}
    \item {Dichte nimmt zu} \\
\end{itemize}
Diese Zusammenhänge liegen an steigender Intensität von London- / Van-der-Waals-kräfte mit steigender Kettenlänge.\\[0,5cm]

Alkane besitzen folgene \textbf{Eigenschaften}:\\
\begin{itemize}
    \item{keine elektrische Leitfähigkeit}
    \item {sie sind unpolar}\\[1cm]
\end{itemize}


\subsection{Halogenalkane}

Halogenalkane sind Alkane, denen durch \textbf{elektrophile Addition}, aus Alkenen, oder durch
\textbf{radikalische Substitution}, aus Alkanen, ein Halogen addiert wurde. \\

\begin{center}
    \chemfig{H-C(-[:90]H)(-[:270]Br)-C(-[:90]H)(-[:270]H)-H} \hspace{1cm} \textbf{Bromethan}\\[0,5cm]
\end{center}

Halogenalkane sind \textbf{lipophil}, ihre \textbf{Siedetemperatur ist höher als bei Alkanen}.
Bei Mehrfachsubstutution / Mehrfachaddition werden die Halogenalkane \textbf{mit steigender Halogenanzahl reaktionsträger}.\\

Eine Nachweisreaktion für Halogenalkane ist die Beilsteinprobe.



\subsection{Kohlenwasserstoffe mit Mehrfachbindungen}
\subsubsection{Alkene und Alkine}

Alkene sind Kohlenwasserstoffe, welche eine Doppelbindung zwischen zwei C-Atomen besitzen.
Sie wie die Alkane benannt, besitzen aber eine \textbf{-en} Endung. Vor dieser Endung wird die Stelle der Mehrfachbindung geschrieben,
z.B.: Pent-2-en.\\

\begin{center}
    \chemfig{C (-[:240]H)(-[:120]H) = C (-[:60]H)(-[:300]H)} \hspace{1cm} \textbf{Ethen}\\[0,5cm]
\end{center}


Alkine bekommen hingegen die Ändung \textbf{-in}.\\

\begin{center}
    \chemfig{H - C ~ C - H} \hspace{1cm} \textbf{Ethin}\\[0,5cm]
\end{center}


\subsubsection{$\pi$ und $\sigma$ Bindungen}

Elektronenpaare in Mehrfachbindungen haben verschiedene Bindungsstärken. $\pi$-Bindungen (Pi-Bindungen) sind schwächer als $\sigma$-Bindungen (Sigma-Bindungen).
Sie befinden sich sozusagen seitlich von der Mitte der beiden Bindungspartner. Bei Dreifachbindungen gibt es zwei $\pi$-Bindungen.\\
Das heißt bei Reaktionen, bei denen eine Mehrfachbindung gespalten werden, wird immer die $\pi$-Bindung gespalten.


\subsubsection{Mehrfachsubstutution}

Wichtig bei Mehrfachsubstutution und Mehrfachaddition ist die Benennung mit cis- und trans-.\\
\begin{center}
    \chemfig{H-C(-[:90]H)(-[:270]\charge{0=\|,180=\|,270=\_}{Br})=C(-[:90]\charge{0=\|,180=\|,90=---}{Br})(-[:270]H)-H} \hspace{1cm} \textbf{Trans-1,2-dibromethen}\\[0,5cm]
    \chemfig{H-C(-[:90]H)(-[:270]\charge{0=\|,180=\|,270=\_}{Br})=C(-[:90]H)(-[:270]\charge{0=\|,180=\|,270=\_}{Br})-H} \hspace{1cm} \textbf{Cis-1,2-dibromethen}\\[0,5cm]
\end{center}

Eine Nachweisreaktion für Mehrfachbindungen ist die Entfärbung von Bromwasser. Der typische Reaktionsmechanismus der Alkene ist die elektrophile Addition.


\subsection{Alkohole}

Alkohole sind Moleküle mit einer O-H-Gruppe (Hydroxygruppe). Sie entstehen durch die
\textbf{nucleophile Substitution} von Halogenalkanen, \textbf{Hydratisierung von Alkenen}








\section{Reaktionsarten}
\subsection{Radikalische Substitution ($S_R$)}

Bei der Radikalischen Substitution wird ein H-Atom eines Kohlenwasserstoffs durch ein Halogen ersetzt.
Sie läuft in drei Schritten ab:\\

\begin{enumerate}
    \item Ein Halogenmolekül wird homolytisch (=in zwei gleiche Teile) in zwei Radikale, mit jeweils einem freien Elektron \textbf{gespalten}.
         \begin{center}
            \schemestart
                \chemfig{\charge{-180=\|,-90=\|,90=\|}{X} - \charge{0=\|,-90=\|,90=\|}{X}} \quad $\xrightarrow{Energie}$ \quad
                \charge{-180=\|,-90=\|,90=\|,0=\.}{X} \quad + \quad \charge{0=\|,-90=\|,90=\|,-180=\.}{X}
            \schemestop
        \end{center}
    \item Eines dieser Halogenradikale greift ein \textbf{H-Atom} des Kohlenwasserstoffs an und \textbf{spaltet} dieses \textbf{ab}. Es bildet sich ein Halogenwasserstoff und eine Kohlenwasserstoffradikal.
            \begin{center}
                \schemestart
                    \charge{-180=\|,-90=\|,90=\|,0=\.}{X} \quad + \quad \chemfig{H - C (-[90]) (-[-90]) (-[0])} \arrow
                    \chemfig{\charge{-180=\|,-90=\|,90=\|}{X} - H} \quad + \quad \chemfig{\charge{-180=\.}{C} (-[90]) (-[-90])-}
                \schemestop
                \end{center}
    \item Es sind mehrere \textbf{Rekombinationsreaktionen} möglich:
        \begin{enumerate}
            \item Es reagieren ein Halogenradikal und ein Kohlenwasserstoffradikal zu einem Halogenkohlenwasserstoff.\\
                \begin{center}
                \schemestart
                    \charge{-180=\|,-90=\|,90=\|,0=\.}{X} \quad + \quad \chemfig{ \charge{-180=\.}{C} (-[90]) (-[-90]) - } \arrow
                    \chemfig{ \charge{-180=\|,-90=\|,90=\|}{X} - C (-[90]) (-[-90]) - }
                \schemestop
                \end{center}
            \item Es reagieren zwei Kohlenwasserstoffradikale miteinander, wobei ein einzelnes Kohlenwasserstoffmolekül aus der Summe beider entsteht.
                \begin{center}
                \schemestart
                    \chemfig{\phantom{R}-\charge{0=\.}{C} (-[90]) (-[-90])} \quad + \quad \chemfig{ \charge{-180=\.}{C} (-[90]) (-[-90]) - } \arrow
                    \chemfig{\phantom{R}-C (-[90]) (-[-90]) - C (-[90]) (-[-90])-}
                \schemestop
                \end{center}
            \item Es reagieren zwei freien Halogenradikale miteinander, wobei ein Halogenmolekül entsteht und ein Kohlenwasserstoffradikal zurückbleibt.
                \begin{center}
                \schemestart
                    \charge{-180=\|,-90=\|,90=\|,0=\.}{X} \quad + \quad \charge{0=\|,-90=\|,90=\|,-180=\.}{X} \arrow
                    \chemfig{\charge{-180=\|,-90=\|,90=\|}{X} - \charge{0=\|,-90=\|,90=\|}{X}}
                \schemestop
                \end{center}
        \end{enumerate}
\end{enumerate}


\subsection{Nucleophile Substitution}

Bei der Nucleophilen Substitution wird ein Halogen eines Halogenkohlenwasserstoffs durch ein Nucleophil ersetzt.\\
Es gibt zwei verschiedene Arten nucleophiler Substitution. Die \textbf{monomolekulare $S_N1$} und die \textbf{bimolekulare $S_N2$}.\\
Bei der $S_N1$ wir zuerst das Halogen abgespalten und ein \ce{X^-}-Ion und ein Kohlenwasserstoffradikal entstehen.
Danach reagiert erst das Nucleophil mit diesem Radikal.\\\\
Bei der $SN_2$ passieren diese zwei Schritte gleichzeitig. Es sind also zeitweise sowohl Halogen, als auch Nucleophil an das selbe C-Atom des Kohlenwasserstoffs gebunden.\\

\subsubsection{$SN_1$}

\begin{enumerate}
    \item Schritt\\
        \begin{center}
        \schemestart
            \chemfig{\phantom{Q}- C (-[90]) (-[-90]) - \charge{0=\|,-90=\|,90=\|}{X}} \arrow
            \chemfig{\phantom{Q}- \ce{C^+} (-[90]) (-[-90])} \quad + \quad \charge{0=\|,-90=\|,90=\|,-180=\|}{\ce{X^-}}
        \schemestop
        \end{center}
    \item Schritt\\
        \begin{center}
        \schemestart
            \chemfig{\phantom{Q}- \ce{C^+} (-[90]) (-[-90])} \quad + \quad \chemfig{\charge{90=\|,-90=\|,-180=\|}{\ce{O^-}} - H} \arrow
            \chemfig{\phantom{Q}- C (-[90]) (-[-90]) - \charge{90=\|,-90=\|}{O} - H}
        \schemestop
        \end{center}
\end{enumerate}

Bei der $SN_1$ bestimmt der erste Schritt die Geschwindigkeit.\\\\

\subsubsection{$SN_2$}

\begin{enumerate}
    \item Schritt\\
        \begin{center}
        \schemestart
            \chemfig{\phantom{Q}- C (-[90]) (-[-90]) - \charge{0=\|,-90=\|,90=\|}{X}} \arrow
            \chemfig{\phantom{Q}- \ce{C^+} (-[90]) (-[-90])} \quad + \quad \charge{0=\|,-90=\|,90=\|,-180=\|}{\ce{X^-}}
        \schemestop
        \end{center}
    \item Schritt\\
        \begin{center}
        \schemestart
            \chemfig{\phantom{Q}- \ce{C^+} (-[90]) (-[-90])} \quad + \quad \chemfig{\charge{90=\|,-90=\|,-180=\|}{\ce{O^-}} - H} \arrow
            \chemfig{\phantom{Q}- C (-[90]) (-[-90]) - \charge{90=\|,-90=\|}{O} - H}
        \schemestop
        \end{center}
\end{enumerate}

\subsection{Elektrophile Addition}
Bei einer elektrophile Addition lagern sich Elektrophile an die Doppelbindungen ungesättigter Moleküle.
Diese ist bimolekular und Zweischrittig.

\begin{enumerate}
    \item Schritt\\
        \begin{center}
        \schemestart
            \chemfig{C (-[:240])(-[:120]) = C (-[:60])(-[:300])} \quad + \quad \chemfig{\charge{90=\|,-90=\|,-180=\|}{\ce{X}}-\charge{90=\|,-90=\|,0=\|}{\ce{X}}}
        \schemestop \\ [5mm]
        Das Halogenmolekül wird von der Doppelbindung \emph{polarisiert}: Es bildet sich der $\pi$-Komplex. \\ [5mm]
        \schemestart
            \chemfig{C (-[:240])(-[:120]) = C (-[:60])(-[:300])} \quad + \quad \chemfig{\charge{90=\|,-90=\|,-180=\|}{\ce{X}}^{\delta -}-\charge{90=\|,-90=\|,0=\|}{\ce{X}}^{\; \delta +}}
        \schemestop \\ [5mm]
        \newpage
        Durch die heterolytische Spaltung des Halogenmoleküls bildet sich der $\sigma$-Komplex. \\ [5mm]
        \schemestart
            \chemfig{[:-30]C(-[:240])(-[:120])*3(-\quad\charge{0=\|,-90=\|,-180=\|}{\ce{X}}^{\;\oplus}-C (-[:60])(-[:300])-)} \quad+\quad \chemfig{\charge{0=\|,90=\|,180=\|,270=\|}{\ce{X}}^{\;\ominus}}
        \schemestop \\ [5mm]
        \end{center}
    \item Schritt\\
        \begin{center}
        Nun greift das Halogenid von der Rückseite an. Es ensteht folgendes Produkt:\\ [5mm]
        \schemestart
            \chemfig{-C ([:90]-\charge{0=\|,90=\|,180=\|}{\ce{X}})  ([:270]-) - C ([:270]-\charge{0=\|,270=\|,180=\|}{\ce{X}})  ([:90]-) -}
        \schemestop \\ [5mm]
        \end{center}
\end{enumerate}
\end{document}